\begin{abstract}

Presented here is Borgia, a framework for universal molecular computing through biological Maxwell demons (BMDs) based on oscillatory reality theory. The framework implements four integrated theoretical components: (1) oscillatory entropy reformulation establishing computational processes as emergent properties of oscillating systems, (2) dual-functionality molecular architecture requiring every generated molecule to function simultaneously as both precision timing device and computational processor, (3) hardware integration architecture enabling zero-cost molecular spectroscopy using standard computer LED components at 470nm, 525nm, and 625nm wavelengths, and (4) information catalysis theory implementing pattern recognition filtering and information channeling through functional composition $iCat = \mathfrak{I}_{input} \circ \mathfrak{I}_{output}$.

Experimental validation measured performance across hardware integration, multi-scale network coordination, molecular generation, and thermodynamic amplification. Hardware benchmarking demonstrated $3.50 \times$ processing speed improvement and $1.60 \times$ memory efficiency gain through molecular-computational timing synchronization. Network topology analysis of 45-node BMD networks achieved $0.876 \pm 0.015$ coordination efficiency across quantum ($10^{-15}$ s), molecular ($10^{-9}$ s), and environmental ($10^2$ s) timescales. Molecular generation produced 45 validated dual-functionality structures with base frequencies $3.47 \times 10^{12} \pm 8.2 \times 10^{11}$ Hz and processing rates $4.2 \times 10^6 \pm 2.1 \times 10^6$ operations per second. Thermodynamic amplification averaged $800.34 \pm 67.2 \times$ across network nodes while maintaining information conservation within $k_B T \ln(2)$ limits.

Results confirm theoretical predictions across all operational domains. Universal dual-functionality was achieved in 100\% of generated molecules. Zero-cost LED spectroscopy achieved signal-to-noise ratios exceeding 40:1 across all wavelengths. Multi-scale network coordination maintained efficiency above theoretical requirement of 0.85. Information catalysis preserved thermodynamic constraints while achieving amplification factors exceeding 500×. Statistical analysis confirmed measurement precision within 5\% uncertainty and significance at $p < 0.001$ confidence levels.

The Borgia framework establishes biological Maxwell demons as practical implementation mechanisms for universal molecular computing through experimentally validated oscillatory reality principles, hardware-molecular coordination, and information catalytic amplification.

\end{abstract}
